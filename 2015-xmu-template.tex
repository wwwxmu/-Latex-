
%%%%%%%%%%%%%%%%%%%%%%%%%%%%%%%%%%%%%%%%%%%%%%%%%%%%%%%%%%%
%版权所有,禁止用于任何商业用途!有任何问题,请与作者联系。
%作者:王玮玮
%Email:www@stu.xmu.edu.cn
%厦门大学自动化系
%2015年5月8日
%%%%%%%%%%%%%%%%%%%%%%%%%%%%%%%%%%%%%%%%%%%%%%%%%%%%%%%%%%%
\documentclass{xmuthesis}
\graphicspath{{figures/}}        % 图片文件路径
\usepackage{setspace}
\usepackage{ragged2e}  %两端对齐 \justifying
%%===参考文献===%%%%%%%%%%%%%%%%%%%%%%%%%%%%%%%%
\bibliographystyle{abbrv}        % 参考文献样式,  plain,unsrt,alpha,abbrv,chinesebst 等等




\begin{document}

%%%%%%% 下面的内容, 据实填空.
\title{厦门大学本科毕业论文~\LaTeX~模板}
\author{王玮玮}                            % 作者名字
\SchoolName{信息科学与技术学院}
\Department{自动化系}
\Major{自动化}
\Grade{2011级}
\StudentNumber{}
\InnerAdviser{}
\TitleOfInnerAdviser{}
\OuterAdviser{}
\TitleOfOuterAdviser{}
\year{2015}
\month{5}
\day{25}

\pdfbookmark[0]{封面}{title}         % 封面页加到 pdf 书签
\maketitle

\frontmatter     % 开始正文之前的部分
\pagenumbering{Roman}              % 正文之前的页码用大写罗马字母编号.
%-----------------------------------------------------------------------------
% !Mode:: "TeX:UTF-8"

%%======中文摘要===========================%
\pagestyle{fancy}
     \fancyhf{}
	 \fancyhead[CO]{\normalfont\small\rmfamily\nouppercase{\leftmark}}
     \fancyhead[CE]{厦门大学本科毕业论文~\LaTeX~模板}
     \fancyfoot[C]{-\,\thepage\,-}
     \renewcommand{\headrulewidth}{0.4pt}
     
     
\begin{cnabstract}
\begin{center}
		\heiti \zihao{3}厦门大学本科毕业论文~\LaTeX~模板 \\[2em]
	\end{center}
\begin{flushleft}
\zihao{4}\heiti[摘要]\quad 
\songti\zihao{-4}此处添加摘要内容
\\[2em]\zihao{4}\heiti[关键词]\quad 
\songti\zihao{-4}关键词一\qquad 关键词二\qquad 关键词三\qquad 
\end{flushleft}
\end{cnabstract}


%%====英文摘要==========================%

\begin{enabstract}
\begin{center}
		\textsf{\zihao{-4}A LATEX Thesis Template for Xiamen University} \\[2em]
	\end{center}
\begin{flushleft}
\arial{\zihao{-4}[Abstract]\quad
\zihao{5}add abstract in this field
\\[2em]\zihao{-4}[Keywords]\quad
\zihao{5}keyword1\qquad keyword2\qquad keyword3\qquad}
\end{flushleft}
\end{enabstract}



    % 加入摘要, 申明.
\pdfbookmark{目录}{toc}
\tableofcontents


%左对齐
\begin{flushleft}
\mainmatter %% 以下是正文


%%%=== 第一章 ========%%%
\chapter{绪论}

\section{课题的研究背景}
\section{国内外开发中使用框架的现状与发展趋势}
\section{论文研究目标和主要工作}
(7)文中图、表应有自明性。图、表名应附相应的英文和必要的中文图注。制图要求:半栏图宽≤7cm,通栏图宽≤16cm;图中曲线粗细应相当于5号宋体字的竖画,坐标线的粗细相当于5号宋体字的横画;图中文字、符号、纵横坐标标目用小五号字;标目采用国家标准的物理量(英文斜体)和单位符号(英文正体)的比表示,如c/molL-1。表格采用“三横线表”,表的内容切忌与图和文字的内容重复。\\
   (8)公式:公式书写应在文中另起一行,居中书写。公式的编号加圆括号,放在公式右边行末,公式和编号之间不加虚线。公式后应注明编号,该编号按章顺序编排。\\
(9)结论二字用四号宋体字,结论内容为小四号宋体字。\\
(10)致谢语三字用四号黑体字,内容为小四号宋体字。\\
   (11)参考文献的著录应执行GB7714-87《文后参考文献著录规则》及《中国学术期刊(光盘版)检索与评价数据规范》规定,采用顺序编码制,在引文处按论文中引用文献出现的先后以阿拉伯数字连续编码,序号置于方括号内。一种文献在同一文中被反复引用者,用同一序号标示,需表明引文具体出处的,可在序号后加圆括号注明页码或章、节、篇名,字体用小五宋体。\\
文后参考文献的著录项目要齐全,其排列顺序以在正文中出现的先后为准;参考文献列表时应以“参考文献:”(左顶格)或“[参考文献]”(居中)作为标识;序号左顶格,用阿拉伯数字加方括号标示;每一条目的最后均以实心点结束。\\
参考文献著录的条目以小于正文的字号编排在文末。文献的著录格式:\\
    ①专著: 作者. 书名[M]. 出版地: 出版社, 出版年.\\
    ②期刊: 作者.题名[J]. 刊名, 出版年, 卷(期): 起止页码.\\
    ③论文集: 作者. 论文集名[C]. 出版地: 出版社, 出版年. 起止页码.\\
    ④学位论文: 作者. 题名[D]. 保存地点: 保存单位, 年.\\
    ⑤专利文献: 专利所有者. 题名[P]. 专利国别: 专利号, 出版日期.\\
    ⑥电子文献: 责任者.电子文献题名[电子文献及载体类型标识]. 电子文献网址. 年-月-日. 文献作者3名以内的全部列出;3名以上则列出前3名,后加"等"(英文加"et al").\\
    参考文献类型,根据GB3469-83《文献类型与文献载体代码》规定,以单字母方式标识:M―专著,C―论文集,N―报纸文章,J―期刊文章,D―学位论文,R―研究报告,S―标准,P―专利;对于专著、论文集中的析出文献采用单字母“A”标识,其他未说明的文献类型,采用单字母“Z”标识。\\
    6、其它要求:\\
   (1)全文内的各章、各节内的标题及段落格式(含顶格或缩进)要一致;\\
   (2)全文内各章的体例要一致,例如,各章(节、目)是否有“导语”;\\
   (3)时间表示:使用“2004年6月”,不能使用“04年6月”或“2004.6”;\\
   (4)标题编号:要符合一般的学术规范,一般不能使用“半括号”,“(一)、”或“(一、)”等不规范用法,标题结束处不能有标点符号;\\
   (5)全文错别字或不规范之处不能超过万分之二。\\
(7)文中图、表应有自明性。图、表名应附相应的英文和必要的中文图注。制图要求:半栏图宽≤7cm,通栏图宽≤16cm;图中曲线粗细应相当于5号宋体字的竖画,坐标线的粗细相当于5号宋体字的横画;图中文字、符号、纵横坐标标目用小五号字;标目采用国家标准的物理量(英文斜体)和单位符号(英文正体)的比表示,如c/molL-1。表格采用“三横线表”,表的内容切忌与图和文字的内容重复。\\
   (8)公式:公式书写应在文中另起一行,居中书写。公式的编号加圆括号,放在公式右边行末,公式和编号之间不加虚线。公式后应注明编号,该编号按章顺序编排。\\
(9)结论二字用四号宋体字,结论内容为小四号宋体字。\\
(10)致谢语三字用四号黑体字,内容为小四号宋体字。\\
   (11)参考文献的著录应执行GB7714-87《文后参考文献著录规则》及《中国学术期刊(光盘版)检索与评价数据规范》规定,采用顺序编码制,在引文处按论文中引用文献出现的先后以阿拉伯数字连续编码,序号置于方括号内。一种文献在同一文中被反复引用者,用同一序号标示,需表明引文具体出处的,可在序号后加圆括号注明页码或章、节、篇名,字体用小五宋体。\\
文后参考文献的著录项目要齐全,其排列顺序以在正文中出现的先后为准;参考文献列表时应以“参考文献:”(左顶格)或“[参考文献]”(居中)作为标识;序号左顶格,用阿拉伯数字加方括号标示;每一条目的最后均以实心点结束。\\
参考文献著录的条目以小于正文的字号编排在文末。文献的著录格式:\\
    ①专著: 作者. 书名[M]. 出版地: 出版社, 出版年.\\
    ②期刊: 作者.题名[J]. 刊名, 出版年, 卷(期): 起止页码.\\
    ③论文集: 作者. 论文集名[C]. 出版地: 出版社, 出版年. 起止页码.\\
    ④学位论文: 作者. 题名[D]. 保存地点: 保存单位, 年.\\
    ⑤专利文献: 专利所有者. 题名[P]. 专利国别: 专利号, 出版日期.\\
    ⑥电子文献: 责任者.电子文献题名[电子文献及载体类型标识]. 电子文献网址. 年-月-日. 文献作者3名以内的全部列出;3名以上则列出前3名,后加"等"(英文加"et al").\\
    参考文献类型,根据GB3469-83《文献类型与文献载体代码》规定,以单字母方式标识:M―专著,C―论文集,N―报纸文章,J―期刊文章,D―学位论文,R―研究报告,S―标准,P―专利;对于专著、论文集中的析出文献采用单字母“A”标识,其他未说明的文献类型,采用单字母“Z”标识。\\
    6、其它要求:\\
   (1)全文内的各章、各节内的标题及段落格式(含顶格或缩进)要一致;\\
   (2)全文内各章的体例要一致,例如,各章(节、目)是否有“导语”;\\
   (3)时间表示:使用“2004年6月”,不能使用“04年6月”或“2004.6”;\\
   (4)标题编号:要符合一般的学术规范,一般不能使用“半括号”,“(一)、”或“(一、)”等不规范用法,标题结束处不能有标点符号;\\
   (5)全文错别字或不规范之处不能超过万分之二。\\(7)文中图、表应有自明性。图、表名应附相应的英文和必要的中文图注。制图要求:半栏图宽≤7cm,通栏图宽≤16cm;图中曲线粗细应相当于5号宋体字的竖画,坐标线的粗细相当于5号宋体字的横画;图中文字、符号、纵横坐标标目用小五号字;标目采用国家标准的物理量(英文斜体)和单位符号(英文正体)的比表示,如c/molL-1。表格采用“三横线表”,表的内容切忌与图和文字的内容重复。\\
   (8)公式:公式书写应在文中另起一行,居中书写。公式的编号加圆括号,放在公式右边行末,公式和编号之间不加虚线。公式后应注明编号,该编号按章顺序编排。\\
(9)结论二字用四号宋体字,结论内容为小四号宋体字。\\
(10)致谢语三字用四号黑体字,内容为小四号宋体字。\\
   (11)参考文献的著录应执行GB7714-87《文后参考文献著录规则》及《中国学术期刊(光盘版)检索与评价数据规范》规定,采用顺序编码制,在引文处按论文中引用文献出现的先后以阿拉伯数字连续编码,序号置于方括号内。一种文献在同一文中被反复引用者,用同一序号标示,需表明引文具体出处的,可在序号后加圆括号注明页码或章、节、篇名,字体用小五宋体。\\
文后参考文献的著录项目要齐全,其排列顺序以在正文中出现的先后为准;参考文献列表时应以“参考文献:”(左顶格)或“[参考文献]”(居中)作为标识;序号左顶格,用阿拉伯数字加方括号标示;每一条目的最后均以实心点结束。\\
参考文献著录的条目以小于正文的字号编排在文末。文献的著录格式:\\
    ①专著: 作者. 书名[M]. 出版地: 出版社, 出版年.\\
    ②期刊: 作者.题名[J]. 刊名, 出版年, 卷(期): 起止页码.\\
    ③论文集: 作者. 论文集名[C]. 出版地: 出版社, 出版年. 起止页码.\\
    ④学位论文: 作者. 题名[D]. 保存地点: 保存单位, 年.\\
    ⑤专利文献: 专利所有者. 题名[P]. 专利国别: 专利号, 出版日期.\\
    ⑥电子文献: 责任者.电子文献题名[电子文献及载体类型标识]. 电子文献网址. 年-月-日. 文献作者3名以内的全部列出;3名以上则列出前3名,后加"等"(英文加"et al").\\
    参考文献类型,根据GB3469-83《文献类型与文献载体代码》规定,以单字母方式标识:M―专著,C―论文集,N―报纸文章,J―期刊文章,D―学位论文,R―研究报告,S―标准,P―专利;对于专著、论文集中的析出文献采用单字母“A”标识,其他未说明的文献类型,采用单字母“Z”标识。\\
    6、其它要求:\\
   (1)全文内的各章、各节内的标题及段落格式(含顶格或缩进)要一致;\\
   (2)全文内各章的体例要一致,例如,各章(节、目)是否有“导语”;\\
   (3)时间表示:使用“2004年6月”,不能使用“04年6月”或“2004.6”;\\
   (4)标题编号:要符合一般的学术规范,一般不能使用“半括号”,“(一)、”或“(一、)”等不规范用法,标题结束处不能有标点符号;\\
   (5)全文错别字或不规范之处不能超过万分之二。\\
%%%=== 第二章 ========%%%
\newpage
\chapter{ThinkPHP~框架开发的关键技术}

\section{框架基本设计思想及项目编译机制}
\section{XAMPP框架构简介}
\subsection{Window操作系统介绍}
\subsection{Apache服务器介绍}
\subsection{MYSQL数据库介绍}
\subsection{PHP介绍}
\subsection{ThinkPHP框架的平台}
(7)文中图、表应有自明性。图、表名应附相应的英文和必要的中文图注。制图要求:半栏图宽≤7cm,通栏图宽≤16cm;图中曲线粗细应相当于5号宋体字的竖画,坐标线的粗细相当于5号宋体字的横画;图中文字、符号、纵横坐标标目用小五号字;标目采用国家标准的物理量(英文斜体)和单位符号(英文正体)的比表示,如c/molL-1。表格采用“三横线表”,表的内容切忌与图和文字的内容重复。\\
   (8)公式:公式书写应在文中另起一行,居中书写。公式的编号加圆括号,放在公式右边行末,公式和编号之间不加虚线。公式后应注明编号,该编号按章顺序编排。\\
(9)结论二字用四号宋体字,结论内容为小四号宋体字。\\
(10)致谢语三字用四号黑体字,内容为小四号宋体字。\\
   (11)参考文献的著录应执行GB7714-87《文后参考文献著录规则》及《中国学术期刊(光盘版)检索与评价数据规范》规定,采用顺序编码制,在引文处按论文中引用文献出现的先后以阿拉伯数字连续编码,序号置于方括号内。一种文献在同一文中被反复引用者,用同一序号标示,需表明引文具体出处的,可在序号后加圆括号注明页码或章、节、篇名,字体用小五宋体。\\
文后参考文献的著录项目要齐全,其排列顺序以在正文中出现的先后为准;参考文献列表时应以“参考文献:”(左顶格)或“[参考文献]”(居中)作为标识;序号左顶格,用阿拉伯数字加方括号标示;每一条目的最后均以实心点结束。\\
参考文献著录的条目以小于正文的字号编排在文末。文献的著录格式:\\
    ①专著: 作者. 书名[M]. 出版地: 出版社, 出版年.\\
    ②期刊: 作者.题名[J]. 刊名, 出版年, 卷(期): 起止页码.\\
    ③论文集: 作者. 论文集名[C]. 出版地: 出版社, 出版年. 起止页码.\\
    ④学位论文: 作者. 题名[D]. 保存地点: 保存单位, 年.\\
    ⑤专利文献: 专利所有者. 题名[P]. 专利国别: 专利号, 出版日期.\\
    ⑥电子文献: 责任者.电子文献题名[电子文献及载体类型标识]. 电子文献网址. 年-月-日. 文献作者3名以内的全部列出;3名以上则列出前3名,后加"等"(英文加"et al").\\
    参考文献类型,根据GB3469-83《文献类型与文献载体代码》规定,以单字母方式标识:M―专著,C―论文集,N―报纸文章,J―期刊文章,D―学位论文,R―研究报告,S―标准,P―专利;对于专著、论文集中的析出文献采用单字母“A”标识,其他未说明的文献类型,采用单字母“Z”标识。\\
    6、其它要求:\\
   (1)全文内的各章、各节内的标题及段落格式(含顶格或缩进)要一致;\\
   (2)全文内各章的体例要一致,例如,各章(节、目)是否有“导语”;\\
   (3)时间表示:使用“2004年6月”,不能使用“04年6月”或“2004.6”;\\
   (4)标题编号:要符合一般的学术规范,一般不能使用“半括号”,“(一)、”或“(一、)”等不规范用法,标题结束处不能有标点符号;\\
   (5)全文错别字或不规范之处不能超过万分之二。\\
%%%=== 第三章 ========%%%
\newpage
\chapter{研究院网站分析与设计}

\section{框架基本设计思想及项目编译机制}
\section{XAMPP框架构简介}
\subsection{Window操作系统介绍}
\subsection{Apache服务器介绍}
\subsection{MYSQL数据库介绍}
\subsection{PHP介绍}
\subsection{ThinkPHP框架的平台}

%%%=== 第四章 ========%%%
\newpage
\chapter{研究院网站实现}

\section{框架基本设计思想及项目编译机制}
\section{XAMPP框架构简介}
\subsection{Window操作系统介绍}
\subsection{Apache服务器介绍}
\subsection{MYSQL数据库介绍}
\subsection{PHP介绍}
\subsection{ThinkPHP框架的平台}

%%%=== 第五章 ========%%%
\newpage
\chapter{结束语及展望}

\section{框架基本设计思想及项目编译机制}
\section{XAMPP框架构简介}
\subsection{Window操作系统介绍}
\subsection{Apache服务器介绍}
\subsection{MYSQL数据库介绍}
\subsection{PHP介绍}
\subsection{ThinkPHP框架的平台}
\backmatter
%%%=== 致谢 ========%%%

% !Mode:: "TeX:UTF-8"
%%%%%%%%%%%%%%%%%%%%%%%%%%%%-------致谢--------%%%%%%%%%%%%%%%%%%%%%%%%%%%%%%%%

\acknowledgement



感谢你, 感谢他和她, 感谢大家.



 

 





 %%%致谢

%%%=== 参考文献 ========%%%

\clearpage\phantomsection
\addcontentsline{toc}{chapter}{参考文献}
\begin{thebibliography}{00}

  \bibitem{r1} 作者. 文章题目 [J].  期刊名, 出版年份,卷号(期数): 起止页码.

  \bibitem{r2} 作者. 书名 [M]. 版次. 出版地:出版单位,出版年份:起止页码.

  \bibitem{r3} 邓建松等, 《\LaTeXe~科技排版指南》, 科学出版社.

  \bibitem{r4} 吴凌云, 《CTeX~FAQ (常见问题集)》, \textit{Version~0.4}, June 21, 2004.

  \bibitem{r5} Herbert Vo\ss, Mathmode, \url{http://www.tex.ac.uk/ctan/info/math/voss/mathmode/Mathmode.pdf}.


\end{thebibliography}


\appendix



\end{flushleft}
\end{document}
